\section{Indexes}

Our first attempt was to add indexes on foreign keys, but in almost all cases, they weren't used, or didn't bring too many advantages, compared with their size.

For the sake of completeness we will report anyway the results (???).

Our second attempt was to add indexes on the attributes used for slicing, so involved in the \verb|WHERE| condition.

\begin{table}[H]
\centering
\begin{tabular}{c|c|c|c|c}
\rowcolor{blue!50} Table & Attribute & Used in Query & Creation time [s] & Index size [MB]\\
\rowcolor{gray!10} lineitem & l\_shipdate & Q1 & 32.43 & 397.54\\ 
\rowcolor{white} lineitem & l\_returnflag & Q1 & 61.83 & 396.46\\ 
\rowcolor{gray!10} lineitem & l\_partkey & Q17 & 46.99 & 429.50\\ 
\rowcolor{white} orders & o\_orderdate & Q10 & 8.10 & \\ 
\end{tabular}\\[0.5cm]
    \caption{Indexes dimensions}
   % \label{tab:costumer_stats}
\end{table}

With these indexes, which are ensured to be used in the execution of the queries, resulted in a total database size of $20.55 GB$.

The execution time of the queries is summarized in the table below.

\begin{table}[H]
\centering
\begin{tabular}{c|c|c}
\rowcolor{blue!50} Query & Mean & Std\\
\rowcolor{gray!10} Q1     &30.009                &1.224\\
\rowcolor{white} Q10    &25.677               &1.799\\
\rowcolor{gray!10} Q14    &23.607               &0.327\\
\rowcolor{white} Q17    &11.232               &0.967 \\
\end{tabular}\\[0.5cm]
    \caption{Execution times of query with indexes}
    \label{tab:idx_exec_time}
\end{table}
