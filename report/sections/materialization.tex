\section{Materialized views}
In this section we will propose some materialized views that aim to improve the execution time of the chosen queries. For the creation of this views we enabled all the hash related operations, since the main purpose of the section is to evaluate performances related to  materialization of some tables, rather than delving into the specifics of this process.

\subsection{Lineitem-part}
\label{ssec:lineitem-part}
In order to improve performances in executing query 14 we decided to create a materialized view as follow:
\begin{sql}
CREATE MATERIALIZED VIEW part_lineitem AS
SELECT
    l_returnflag,
    l_linestatus,
    l_quantity,
    l_extendedprice,
    l_discount,
    l_tax,
    l_shipdate,
    l_partkey,
    p_partkey,
    p_brand,
    p_container,
    SUBSTRING(p_type FROM 1 FOR 5) AS p_type_prefix,
    0.2 * AVG(l_quantity) OVER (PARTITION BY l_partkey) AS avg_quantity
FROM
    lineitem l
JOIN
    part p ON l.l_partkey = p.p_partkey;
\end{sql}

\textbf{Statistics of this view:}
\begin{itemize}
    \item Required time to create the view 310.952 seconds.
    \item Size of the view: 6.43 GB.
\end{itemize}

We rewrote the query 14 and 17 in order to exploit the materialized views.


Materializing the join operation of query 14 we expected to have a lower execution time with respect to the baseline, but this did not happened. By observing the output of the \emph{EXPLAIN ANALYZE} function we understood that the problem with the lack of gain in performance is that the optimizer performs a sequential scan of the lineitem table, since there is no index on shipdate.

The greatest improvement in using this materialization was expected to be in query 17. Indeed the execution time dropped to 15.265 seconds (on a single run), which is a sensible gain in terms of performances. 

\subsection{Costumer-orders-lineitem-nation}
\label{ssec:costumer-orders-lineitem-nation}
In order to spare the most time of the joins in query 10 we created a bigger materialized view.
\begin{sql}
CREATE MATERIALIZED VIEW customer_order_lineitem_nation AS
SELECT
    c.c_custkey,
    c.c_name,
    c.c_acctbal,
    n.n_name,
    c.c_address,
    c.c_phone,
    c.c_comment,
    l.l_returnflag,
    l.l_discount,
    l.l_extendedprice,
    o.o_orderdate

FROM
    customer c
JOIN
    orders o ON c.c_custkey = o.o_custkey
JOIN
    lineitem l ON l.l_orderkey = o.o_orderkey
JOIN
    nation n ON c.c_nationkey = n.n_nationkey;
\end{sql}

\textbf{Statistics of this view:}
\begin{itemize}
    \item Required time to create the view 437.107 seconds.
    \item Size of the view: 12.63 GB.
\end{itemize}


We tested the performances of query 10 rewritten using this materialization, but there was no gain in efficiency, other than the fact that this materialization was made ad-hoc for this single query.

\subsection{Lineitem-part-orders}
As a last option we opted for a mixed approach creating a view which is neither as big nor as specific as the previous ones.

\begin{sql}
CREATE MATERIALIZED VIEW part_lineitem_order AS
SELECT
    l_returnflag,
    l_linestatus,
    l_quantity,
    l_extendedprice,
    l_discount,
    l_tax,
    l_shipdate,
    l_partkey,
    p_partkey,
    p_brand,
    p_container,
    SUBSTRING(p_type FROM 1 FOR 5) AS p_type_prefix,
    0.2 * AVG(l_quantity) OVER (PARTITION BY l_partkey) AS avg_quantity,
    o_orderkey,
    o.o_custkey,
    o.o_orderdate
FROM
    lineitem l
JOIN
    part p ON l.l_partkey = p.p_partkey
JOIN
    orders o ON l.l_orderkey = o.o_orderkey;
\end{sql}

\textbf{Statistics of this view:}
\begin{itemize}
    \item Required time to create the view 366.508 seconds.
    \item Size of the view: 6.93 GB.
\end{itemize}

As this is the expected best materialization, we report the modified queries that use it.

\begin{sql}[Query 10]
SELECT
    c_custkey,
    c_name,
    SUM(l_extendedprice * (1 - l_discount)) AS revenue,
    c_acctbal,
    n_name,
    c_address,
    c_phone,
    c_comment
FROM
    part_lineitem_order
    JOIN customer c ON c.c_custkey = o_custkey
    JOIN nation n ON c.c_nationkey = n.n_nationkey
WHERE
    o_orderdate >= DATE '1993-10-01'
    AND o_orderdate < DATE '1993-10-01' + INTERVAL '3' MONTH
    AND l_returnflag = 'R'
GROUP BY
    c_custkey,
    c_name,
    c_acctbal,
    c_phone,
    n_name,
    c_address,
    c_comment
ORDER BY
    revenue DESC;
\end{sql}

\begin{sql}[Query 14]
SELECT
    100.00 * SUM(CASE
        WHEN p_type_prefix LIKE 'PROMO'
        THEN l_extendedprice * (1 - l_discount)
        ELSE 0
    END) / SUM(l_extendedprice * (1 - l_discount)) AS promo_revenue
FROM
    part_lineitem_order
WHERE
    l_shipdate >= DATE '1995-09-01'
    AND l_shipdate < DATE '1995-09-01' + INTERVAL '1' MONTH;
\end{sql}

\begin{sql}[Query 17]
SELECT
    SUM(l_extendedprice) / 7.0 AS avg_yearly
FROM
    part_lineitem_order
WHERE
    p_brand = 'Brand#23'
    AND p_container = 'MED BOX'
    AND l_quantity < avg_quantity;
\end{sql}

In this case we performed a complete benchmark, and the results are reported in the table below \ref{tab:materialize}.

\begin{table}[H]
\centering 
\begin{tabular}{c|c|c} 
\rowcolor{blue!50} Query & Mean & Std\\
\rowcolor{gray!10} Q1 &24.770 &2.034\\
\rowcolor{white} Q10 &50.135 &59.566\\
\rowcolor{gray!10} Q14 &20.677 &0.353\\
\rowcolor{white} Q17 &20.465 &0.655\\
\end{tabular}\\[0.5cm] 
\caption{Execution times of query with materialized views} 
\label{tab:materialize} 
\end{table}



