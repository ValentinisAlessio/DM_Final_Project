
\section{Query schemas}

The assignment consists in using TPC-Benchmark H to test and optimize a set of four queries, using indexes, materialized views, a mixed approach of the two and fragmentation.

The set of queries selected for the assignment are Q1, Q10, Q14, Q17 of the Official Documentation. An overall view on the description and SQL implementation is given below.

\subsection{Query 1}
\textbf{Brief description}
The Pricing Summary Report Query provides a summary pricing report for all lineitems shipped as of a given date.
The date is within 60 - 120 days of the greatest ship date contained in the database. The query lists totals for
extended price, discounted extended price, discounted extended price plus tax, average quantity, average extended
price, and average discount. These aggregates are grouped by RETURNFLAG and LINESTATUS, and listed in
ascending order of RETURNFLAG and LINESTATUS. A count of the number of lineitems in each group is
included.

\textbf{Functional definition}
\begin{sql}[ ]
SELECT
    l_returnflag,
    l_linestatus,
    SUM(l_quantity) AS sum_qty,
    SUM(l_extendedprice) AS sum_base_price,
    SUM(l_extendedprice * (1 - l_discount)) AS sum_disc_price,
    SUM(l_extendedprice * (1 - l_discount) * (1 + l_tax)) AS sum_charge,
    AVG(l_quantity) AS avg_qty,
    AVG(l_extendedprice) AS avg_price,
    AVG(l_discount) AS avg_disc,
    COUNT(*) AS count_order
FROM
    lineitem
WHERE
    l_shipdate <= DATE '1998-12-01' - INTERVAL '[DELTA]' DAY
GROUP BY
    l_returnflag,
    l_linestatus
ORDER BY
    l_returnflag,
    l_linestatus;
\end{sql}

For a matter of simplicity we decided to take as slicing values the ones proposed by the validation paragraph in the official documentation. (So

\verb|'[DELTA]' = '90'|)

\subsection{Query 10}
\textbf{Brief description}
The Returned Item Reporting Query finds the top 20 customers, in terms of their effect on lost revenue for a given
quarter, who have returned parts. The query considers only parts that were ordered in the specified quarter. The
query lists the customer's name, address, nation, phone number, account balance, comment information and revenue
lost. The customers are listed in descending order of lost revenue. Revenue lost is defined as
sum(l\_extendedprice*(1-l\_discount)) for all qualifying lineitems.

\textbf{Functional definition}
\begin{sql}
SELECT
    c_custkey,
    c_name,
    SUM(l_extendedprice * (1 - l_discount)) AS revenue,
    c_acctbal,
    n_name,
    c_address,
    c_phone,
    c_comment
FROM
    customer,
    orders,
    lineitem,
    nation
WHERE
    c_custkey = o_custkey
    AND l_orderkey = o_orderkey
    AND o_orderdate >= DATE '[DATE]'
    AND o_orderdate < DATE '[DATE]' + INTERVAL '3' MONTH
    AND l_returnflag = 'R'
    AND c_nationkey = n_nationkey
GROUP BY
    c_custkey,
    c_name,
    c_acctbal,
    c_phone,
    n_name,
    c_address,
    c_comment
ORDER BY
    revenue DESC;
\end{sql}

For a matter of simplicity we decided to take as slicing values the ones proposed by the validation paragraph in the official documentation. (So

\verb|'[DATE]' = '1993-10-01'|)

\subsection{Query 14}
\textbf{Brief description}
The Promotion Effect Query determines what percentage of the revenue in a given year and month was derived from 
promotional parts. The query considers only parts actually shipped in that month and gives the percentage. Revenue 
is defined as (l\_extendedprice * (1-l\_discount)).


\textbf{Functional definition}
\begin{sql}
SELECT 
    100.00 * SUM (CASE WHEN p_type like 'PROMO%' 
                  THEN l_extendedprice*(1-l_discount) 
                  ELSE 0 END) / SUM(l_extendedprice * (1 - l_discount)) 
    AS promo_revenue
FROM 
    lineitem, 
    part
WHERE 
    l_partkey = p_partkey
    AND l_shipdate >= date '[DATE]'
    AND l_shipdate < date '[DATE]' + interval '1' month;

\end{sql}

For a matter of simplicity we decided to take as slicing values the ones proposed by the validation paragraph in the official documentation. (So \verb|'[DATE]' = '1995-09-01'|)


\subsection{Query 17}

\textbf{Brief description}
The Small-Quantity-Order Revenue Query considers parts of a given brand and with a given container type and 
determines the average lineitem quantity of such parts ordered for all orders (past and pending) in the 7-year database. What would be the average yearly gross (undiscounted) loss in revenue if orders for these parts with a quantity 
of less than 20 \% of this average were no longer taken?

\textbf{Functional definition}
\begin{sql}
SELECT
    SUM(l_extendedprice) / 7.0 AS avg_yearly
FROM 
    lineitem, 
    part
WHERE 
    p_partkey = l_partkey
    AND p_brand = '[BRAND]'
    AND p_container = '[CONTAINER]'
    AND l_quantity < ( 
                        SELECT 
                            0.2 * AVG(l_quantity)
                        FROM 
                            lineitem
                        WHERE 
                            l_partkey = p_partkey
                      );

\end{sql}

For a matter of simplicity we decided to take as slicing values the ones proposed by the validation paragraph in the official documentation. (So 

\verb|'[BRAND]' = 'Brand#23'| and \verb|'[CONTAINER]' = 'MED BOX'|)