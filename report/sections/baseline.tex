\section{Baseline}

We decided to test the execution time of queries without additional indexes or views, other than the keys suggested by the documentation, and to use it as a baseline for further improvement in the management of the queries.

In order to keep the results as similar as possible to the theoretical case, we decided to disable any kind of hash-related operations, that in PostgreSQL are very much used and optimized, and can alter the improvement of the execution time when introducing queries.

The results of the tests are reported in the plot at table \ref{tab:baseline_exec_time}.


The execution times for Query 17 were not reported in the table because it was not feasible to execute it within reasonable time frames to obtain mean and standard deviation values. Since after two hours of computation it had still not produced a result, we decided to estimate empirically the possible execution times. We ran the query on a limited dataset, attempting to understand the relationship between the number of rows in the table and execution times. Given that executing the query on data related to one month, or 103.000 rows, requires a cost of 144.035 operations and takes 2 seconds, and running the query on data related to three months, or 8.814 rows, requires a cost of 6.091.570 and takes 173 seconds, we reasonably assumed that executing the query on the entire table composed of 60 million rows, with a cost of 117.789.668.871 operations, would take significantly longer than what we had available. Additionally, since the purpose of data warehousing is to support the decision-making, this query without optimization becomes useless and thus reporting its cost would be irrelevant if it exceeds a couple of hours.

\begin{table}[H]
\centering 
\begin{tabular}{c|c|c} 
\rowcolor{blue!50} Query & Mean [s] & Std [s] \\ 
\rowcolor{gray!10} Q1  & 41.778   & 1.412   \\
\rowcolor{white}   Q10 & 33.077   & 1.634   \\
\rowcolor{gray!10} Q14 & 28.253   & 1.239   \\
\rowcolor{white}   Q17 & N/A      & N/A     \\
\end{tabular}\\[0.5cm] 
\caption{Execution times of query} 
\label{tab:baseline_exec_time} 
\end{table}

[MAYBE ADD MEDIAN]