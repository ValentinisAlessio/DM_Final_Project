\section{Mixed approach}
In this section we explored a mixed approach with both materialization and indexing.
As we already sapred the join conditions, we will just consider indexes that are useful for slicing.
Remember that the chosen materialized view results to be \textit{Lineitem-part-orders}, but for the sake of completeness, we reported also indexing on all the materialization attempts, in order to record possible time improvement. 

For every materialization we will report a table with the introduced indexes and their creation time and size.

\subsection{Lineitem-part}

\begin{table}[H]
\centering
\begin{tabular}{c|c|c}
\rowcolor{blue!50}  Attribute     & Creation time [s] & Index size [MB]\\
\rowcolor{gray!10} l\_shipdate     & 30.16          & 397.55 \\ 
\rowcolor{white}   p\_brand        & 58.95          & 403.14 \\
\rowcolor{gray!10} p\_container    & 60.34          & 403.15 \\ 
\end{tabular}\\[0.5cm]
    \caption{Indexes dimensions}
   % \label{tab:costumer_stats}
\end{table}
Using this indexes the total size of tables is: 7.60 GB.

Remember that this materialization is useful mainly for queries 14 and 17, so the indexes were placed on the suitable slicing attributes.

\subsection{Costumer-orders-lineitem-nation}

\begin{table}[H]
\centering
\begin{tabular}{c|c|c}
\rowcolor{blue!50}  Attribute     & Creation time [s] & Index size [MB]\\
\rowcolor{gray!10} o\_orderdate     & 58.30       & 397.51 \\ 
\rowcolor{white}   l\_returnflag    & 50.62       & 396.46 \\
\end{tabular}\\[0.5cm]
    \caption{Indexes dimensions}
   % \label{tab:costumer_stats}
\end{table}
Using this indexes the total size of tables is: 13.41 GB.

Recalling that this materialization is pretty useful only for one query, we exploited indexing only for the proper attributes, but given the resulting size of the table, we furthermore opted for dropping this big materialization in favor of smaller ones.

\subsection{Lineitem-part-orders}

\begin{table}[H]
\centering
\begin{tabular}{c|c|c}
\rowcolor{blue!50}  Attribute       & Creation time [s] & Index size [MB]\\
\rowcolor{gray!10} l\_shipdate      & 47.12          & 397.55 \\ 
\rowcolor{white} o\_orderdate       & 44.72          & 397.51 \\ 
\rowcolor{gray!10} l\_returnflag    & 47.30          & 396.46 \\
\rowcolor{white}   p\_brand         & 49.68          & 403.14 \\
\rowcolor{gray!10} p\_container     & 59.57          & 403.15 \\ 
\end{tabular}\\[0.5cm]
    \caption{Indexes dimensions}
   % \label{tab:costumer_stats}
\end{table}

Using this indexes the total size of tables is: 8.11 GB.

As this materialization resulted to be useful for three of the four proposed queries, we added indexes for all the useful slicing attributes involved in queries 10, 14, 17.

Recalling the fact that the chosen final materialization is the \textit{lineitem-part-orders}, with all the indexes the final database size is of 32.23 GB.

We reported in table \ref{tab_mixed} the results of testing the queries with the described structure.



\begin{table}[H] 
\centering \begin{tabular}{c|c|c} 
\rowcolor{blue!50} Query & Mean & Std\\
\rowcolor{gray!10} Q1 &23.184 &1.360\\
\rowcolor{white} Q10 &34.227 &4.548\\
\rowcolor{gray!10} Q14 &24.086 &1.504\\
\rowcolor{white} Q17 &1.587 &0.103\\
\end{tabular}\\[0.5cm] 
\caption{Execution times of query with both materialized views and indexes} 
\label{tab:mixed} 
\end{table}